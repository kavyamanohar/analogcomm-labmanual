\chapter[AM Detection with Automatic Gain Control]{AM Detection with Automatic Gain Control}
\section*{Aim}
To demodulate the message content from AM signal. Also detect the automatic gain control signal from the received AM signal.
\section*{Theory}
A simple AM demodulator is a diode envelope detector as discussed in  Chapter \ref{AMDetection}.
\section*{Design}After the envelope detector, a properly designed low pass filter is added to filter out the high frequency carrier and to contain the low frequency modulating signal. This signal contains a dc level also which can be used for automatic Gain Control (AGC) for the IF amplifier stages of a superhetrodyne receiver.

Let the carrier frequency be $f_c=455\ kHz$ and maximum modulating signal frequency be $f_m=10\ kHz$

Inorder to design a lowpass filter with upper cutoff frequency 10 kHz,
\begin{equation}
f_H=\frac{1}{2\pi R_dC_d}
\end{equation}
\begin{equation}
10\ kHz=\frac{1}{2\pi R_dC_d}
\end{equation}
\noindent Select $C_d=\ 0.001 \mu F$. Then $R_d=\ 16.1k\Omega$.
Choose $R_d=\ 15k\Omega \ or\ 22k\Omega$ standard resistor values.

Make a $\pi$ filter using these $R_d$ and $C_d$ values. This completes the envelope detector part.
\paragraph{AGC Circuit:} The demodulated output voltage is having a modulating signal and a dc offset voltage. The dc offset voltage is proportional to the strength of the modulated signal received by the receiver in a transmission reception system, which inturn is proportional to the strength(amplitude) of the carrier.

From the detected output this dc voltage is separated by a LPF to get the required AGC feedback voltage. In superheterodyne receivers this AGC voltage is fed back to the IF stage of the receiver to increase or decrease the gain of the amplifier in such a way to get a constant output from the receiver. The AGC lowpass filter $R_a$ and $C_a$ in such a way as to eliminate full ac from the output and get a pure dc AGC voltage. 
Hence assuming a cutoff frequency of 10 Hz,
\begin{equation}
10 Hz= \frac{1}{2\pi R_aC_a}
\end{equation}
\noindent Assuming $C_a=1 \mu F$, we get $R_a=15 k \Omega$

\section*{Circuit Diagram}
\section*{Procedure}
\section*{Observation}
\section*{Result}
