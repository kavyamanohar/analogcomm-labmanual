\chapter[AM generation using IFT]{AM generation using IFT}

\section*{Aim}
To design and set-up  an AM generator using BJT and IFT and measure the modulation index from the observed output waveform.

\section*{Theory}
Any amplifier can be converted into a sinusoidal oscillator if Barkhausen conditions are satisfied. So tuned amplifier in chapter \ref{iftamplifier} can be converted ito a high frequency oscillator for generating carrier wave by providing a positive feedback after removing  the input and the load.

Inorder to obtain the feedback signal the terminal-1 of the IFT primary coil is used. It is $180^{\circ}$ out of phase with the signal at collector, ie. terminal-2 of IFT primary winding. The collector signal is already $180^{\circ}$ out of phase with the signal at base of BJT. Thus the feed back signal from terminal-1 of the IFT to the base of BJT is in phase with the signal at the base. The feedback capacitor is chosen to be low to avoid additional phase shift.  

The circuit now works as an oscillator generating a signal of frequency of around 455 kHz.Its amplitude, $E_c$ can be slightly adjusted by varying the potentiometer connected in series with the emitter resistance and frequency, $f_c$ by tuning the IFT. This circuit can be 

\section*{Design}

\section*{Circuit Diagram}

\section*{Procedure}
\section*{Observation}
\section*{Result}

